\documentclass[handout]{ximera}
%% handout
%% space 
%% newpage
%% numbers
%% instructornotes

%% You can put user macros here
%% However, you cannot make new environments

\usepackage[letterpaper, total={6in, 8in}]{geometry}
\graphicspath{{./}{firstExample/}{secondExample/}}

\usepackage{tikz}
\usepackage{tkz-euclide}
\usetkzobj{all}

\tikzstyle geometryDiagrams=[ultra thick,color=blue!50!black]
 %% we can turn off include when making a master document


\outcome{Understand a second example of the Ximera style.}
\outcome{See how to include graphics.}

\title{Syllabus for Math R300 \hfill
Spring 2017 \hfill \,}
\author{Oguz Kurt}
%\email{oguzkurt@gmail.com}
\date{Spring 2017}
\begin{document}
\begin{abstract}
Please, refer to this page for various information on this course.
\end{abstract} 
\maketitle

\section*{Course Information}
\begin{tabular}[c]{lll}
Class Times & : &  Monday - Wednesday 1:00 PM -- 2:15 PM  \\
Location & : &  825  \\
Instructor & : &  Oguz Kurt  \\
E-mail & : &  {\link[\textbf{oguzkurt@gmail.com}]{mailto:oguzkurt@gmail.com}} \\
Office & : & 831 \\
Office Hours & : & MW 11:00 AM -- 12:30 PM (and by appointment.) 
\end{tabular}

\section*{Prerequisite(s)}
None
\section*{Course Description}

This course reviews basic arithmetic skills and pre-algebra, and elementary algebra topics that are required for the College Algebra course.

\section*{Textbook}

We will be using two free, online textbooks Prealgebra and College Algebra (first few chapters) prepared by the not-for-profit 
{\link[\textbf{OpenStax}]{https://www.openstax.org} collobaration within Rice University. You may support OpenStax by simply going to their webpage and donating as low as \$5.

\subsection*{Textbook Copyright and Attribution}

The OpenStax College name, OpenStax College logo, OpenStax College book covers, OpenStax CNX name, and OpenStax CNX logo are not subject to the creative commons license and may not be reproduced without the prior and express written consent of Rice University. For questions regarding this license, please contact \link[\textbf{partners@openstaxcollege.org}]{mailto:partners@openstaxcollege.org}.


\section*{Instructional Materials}

Handouts will frequently be provided. The students can use a calculator only for checking their work. You are {\bf NOT} allowed to use calculators during any in-class testing via quizzes or exams.  The students might be asked to bring a notebook or tablet computer to class only to be used for group study purposes. Otherwise, the use of cell phones and, tablet or notebook computers is strictly prohibited. 

\section*{Instructional Methods}

Lectures, group study, online study materials.

\section*{Course Website(s)}

General course information, announcements, lecture materials will be delivered through the North American University Online system – NAU Moodle via \link[\textbf{http://www.na.edu}]{http://www.na.edu}. The students can also monitor their grades. The login username and password are as same as the student computer account login information. The students are required to fill out their profile information in first login. Please visit IT Department if you have trouble in signing in.

The textbooks for this course may be reached online at 
\\ 
\link[\textbf{https://openstax.org/details/books/prealgebra}]{ https://openstax.org/details/books/prealgebra}
\\ 
\link[\textbf{https://openstax.org/details/books/college-algebra}]{https://openstax.org/details/books/college-algebra}

The HW and some study materials will be provided online at 
\\ 
\link[\textbf{http://ximera.osu.edu/course/oguzkurt/PreAlgebraCourse}]{http://ximera.osu.edu/course/oguzkurt/PreAlgebraCourse} 
\\
Note that you will need to sign-up to receieve HW credit. Please, use a gmail account that reflects your actual name so that I will not have to ask you which email is yours.

\section*{Homework and Expectations}

Students are expected to spend approximately six (6) hours a week, on average, completing homework assignments in order to achieve the learning objectives for this 15 week lecture course. This meets the Federal Government’s expectation of two hours of homework for each hour of lecture. 

Homework will be {\it regularly} assigned online via \link[\textbf{our course page with the Ximera Project at The Ohio State University}]{http://ximera.osu.edu/course/oguzkurt/PreAlgebraCourse}. Students must also keep a small notebook for HW purposes only. They are expected to {\it tediously} write the full solution for each online HW problem in it. This notebook will be checked regulary to ensure that you are learning this course properly. 

\section*{Exams}

There will be 4 in class midterms that will cover the topics of this course incrementally. Tests are closed book. Calculators are {\bf NOT} allowed. See the course outline at the end of this document for tentative exam dates. The lowest score will be dropped at the end of the semester. If you miss one test, it will count as the dropped midterm score. Otherwise, no make-up exams will be given unless a very clear, official documentation of your absence is provided.

There will also be a {\bf cumulative} final exam at the end of the semester.

\section*{Academic Honesty}

Each student assumes the responsibilities of being a member of the NAU academic community.  All acts of plagiarism are not tolerated including: cheating, claiming one’s work as their own, fabrication and helping one to commit any of these acts.  Any violations of academic honesty will receive strict disciplinary action, which can include suspension and even expulsion from NAU.  

\section*{Accommodatations}

Students that require any accommodation (such are students with disabilities, religious conflicts, etc…) should notify the instructor as early as possible and accommodations will be made on an individual basis in adherence with the regulations outlined in the Student Handbook.

\section*{Assessment Criteria \& Methods of Evaluating Students}

\begin{tabular}[c]{lllcccl}
3 Midterms  & : & 45\% & \,\hspace{2cm} &  & & \\  
Final & : & 30\% & \hspace{2cm} & & & \\  
Homework & : & 15\% & \hspace{2cm} & & & \\  
Quiz & : & 10\% & \hspace{2cm} & {\bf PASS}& :& 60\% or above, or 90\% from final exam.\\  
Attendance & : & 5\% & \hspace{2cm} & {\bf FAIL}& : & Below 60\%\\  
\end{tabular}


\section*{Course Outline} 

To be added online

%\begin{instructorNotes}
%  Here we see a multi-part question.
%\end{instructorNotes}

%\begin{instructorIntro}
%  This should tell me something
%\end{instructorIntro}


\end{document}

