\documentclass{ximera}
%% You can put user macros here
%% However, you cannot make new environments

\usepackage[letterpaper, total={6in, 8in}]{geometry}
\graphicspath{{./}{firstExample/}{secondExample/}}

\usepackage{tikz}
\usepackage{tkz-euclide}
\usetkzobj{all}

\tikzstyle geometryDiagrams=[ultra thick,color=blue!50!black]

\title{Exponents\\ Homework 6}
\author{Oguz Kurt}
\begin{abstract}
    Add, subtract, multiply, divide polynomials.
\end{abstract}

\outcome{Identify, distinguish natural, whole numbers and integers.}
\outcome{Use natural, whole numbers and integers in operations.}
\outcome{Learn the impact of different operations on natural, whole numbers and integers.}

\begin{document}
\maketitle

\section*{Syntax}
To enter {\Large $a^b$}, you need to use {\Large \color{red}\verb|a^b|}.

\section*{Exponential Laws}

\begin{itemize}
    \item For $n>0$, $a^n=\underbrace{a\cdot a \cdot \ldots \cdot a}_{Multiply~ n ~times} $
    \begin{example}
        $2^5=\answer{32}$
    \end{example}
    \item For $a\neq 0$, $a^0=1$
    \begin{example}
        $5^0=\answer{1}$
    \end{example}
    \item $a^n\cdot a^m=a^{n+m}$
    \begin{example}
        $3^4\cdot 3^3=3^{\answer{7}}=\answer{\sage{3^7}}$
    \end{example}

    \item $\left(a^n\right)^m=a^{n\cdot m}$
    \begin{example}
        $\left(2^5\right)^2=2^{\answer{10}}=\answer{1024}$    
    \end{example}
    \item $a^n\cdot b^n =(a\cdot b)^n$
    \begin{example}
        $2^7\cdot 5^7=\answer{10}^{\answer{7}}=\answer{\sage{10^7}}$
    \end{example}
    \item $a^{-1}=\frac{1}{a}$
    \begin{example}
        $4^{-1}=\answer{\frac{1}{4}}$
    \end{example}
    \begin{example}
        $3^{-4}=(3^4)^{\answer{-1}}=\frac{1}{\answer{81}}$
    \end{example}
    \item $\frac{a^n}{a^m}=a^{n-m}$
    \begin{example}
        $\frac{7^9}{7^3}=\left(\answer{7}\right)^{\answer{9-3}}=\answer{\sage{7^6}}$
    \end{example}
    \item $\frac{a^n}{b^n}=\left( \frac{a}{b}\right)^n$
    \begin{example}
        $\frac{6^9}{2^9}=\left(\answer{\frac{6}{2}}\right)^{\answer{9}}=\answer{\sage{3^9}}$    
    \end{example}
\end{itemize}


%\begin{shuffle}[1]
\begin{problem}
$2^3 \cdot 2^8 =\answer{\sage{2^11}} $
\end{problem}

\begin{problem}
$5^3 \cdot 5^8 \cdot 5 = \answer{\sage{5^12}} $ 
\end{problem}

\begin{problem}
$(a^3 b^2)(a^3 b^5) = a^{\answer{6}}\cdot b^{\answer{7}}$
\end{problem}

\begin{problem}
$\displaystyle \frac{x^8}{x^2} = x^{\answer{6}} $
\end{problem}

\begin{problem}
$\displaystyle \frac{7^9}{7^7} = \answer{7}^{\answer{2}}=\answer{49} $
\end{problem}

\begin{problem}
 $\displaystyle \frac{4p^5q^7}{6p^2q} = \answer{4/6}\cdot p^{\answer{5-2}} \cdot q^{\answer{7-1}} $
\end{problem}

\begin{problem}
 $1948^0 = \answer{1}$ 
\end{problem}

\begin{problem}
 $(-9)^0 = \answer{1}$ 
\end{problem}

\begin{problem}
$-9^0 = \answer{-1}$ 
\end{problem}

\begin{problem}
$(2^3)^4 = \answer{2}^{\answer{12}}$
\end{problem}

\begin{problem}
$(m^2)^5 = m^{\answer{10}}$ 
\end{problem}

\begin{problem}
$(3a)^4 = \answer{81}\cdot a^{\answer{4}}$
\end{problem}

\begin{problem}
$(-5x^4)^2 = \answer{25}\cdot x^{\answer{8}}$ 
\end{problem}

\begin{problem}
$(a^7 b)^2(a^3 b^4) = a^{\answer{17}}\cdot b^{\answer{6}}$
\end{problem}

{\color{red} {\bf WARNING!!!} Rule applies to products, not sums or differences:
$$ (5 + x)^2 \neq 5^2 + x^2 $$
}

\begin{problem}
$\displaystyle \left(\frac{x}{5} \right )^2 = \frac{\answer{x^2}}{\answer{25}}$
\end{problem}

\begin{problem}
$\displaystyle\left(\frac{5}{a^4} \right )^3 = \frac{\answer{125}}{\answer{a^{12}}}$
\end{problem}

\begin{problem}
$\displaystyle\left(\frac{3a^4}{b^3} \right )^2 = \frac{\answer{9a^8}}{\answer{b^6}}=\answer{9}\cdot a^{\answer{8}}\cdot b^{\answer{-6}}$
\end{problem}

\begin{problem}
$\displaystyle m^{-3}= \frac{\answer{1}}{\answer{m^3}}$
\end{problem}

\begin{problem}
$\displaystyle (-3)^{-2}= \answer{\frac{1}{9}}$
\end{problem}

\begin{problem}
$a b^{-1} = \answer{\frac{a}{b}}$
\end{problem}

\begin{problem}
$\displaystyle t^5 \cdot t^{-2} = t^{\answer{3}}$
\end{problem}

\begin{problem}
$\displaystyle (5x^2y^{-3})^{4}= \answer{5^4}\cdot x^{\answer{8}}\cdot y^{\answer{-12}}=\frac{\answer{625x^8}}{\answer{y^{12}}}$
\end{problem}

\begin{problem}
$\displaystyle \frac{-10 x^{-3}y}{5x^2y^5} = \answer{-2}\cdot x^{\answer{-5}}\cdot y^{\answer{-4}}=\frac{\answer{-2}}{\answer{x^5y^4}}$
\end{problem}

\end{document}
