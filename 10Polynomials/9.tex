\documentclass{ximera}
%% You can put user macros here
%% However, you cannot make new environments

\usepackage[letterpaper, total={6in, 8in}]{geometry}
\graphicspath{{./}{firstExample/}{secondExample/}}

\usepackage{tikz}
\usepackage{tkz-euclide}
\usetkzobj{all}
\usetikzlibrary{arrows}
\tikzstyle geometryDiagrams=[ultra thick,color=blue!50!black]

%\usepackage[dvipsnames]{xcolor}

\title{Polynomial Factorization \\ Homework 9}
\author{Oguz Kurt}
\begin{abstract}
    Factorize polynomials.
\end{abstract}
\usepackage{booktabs}

\newcommand*{\Ph}{\hphantom{{}+{}}}%
\begin{document}
\maketitle
\js{10+1}
In this homework, we will introduce factorization which is the opposite of multiplication of polynomials. To be able to factor polynomials, you should have a very good understanding of the multiplications methods we learned previously. \\

\section{Factorization}

{\bf Factoring:} Finding an equivalent expression that is a product of polynomials. This equivalent expression is called factorization, each polynomial is called a factor. If a polynomial cannot be factorized, we say that the polynomial is prime.\\

Remember that the same concept is discussed with numbers:\\

\begin{problem}
Factorize 120 into its prime factors.\\
\end{problem}
 
\subsection{Factoring monomials}

To factor a monomial find two monomials whose product is equivalent to the original one. The answer might not be unique.\\

\begin{problem} 
Factorize $15 x^3$.\\
\end{problem}

 
\subsection{Factoring Polynomials Whose Terms Have a Common Factor}

To factor a polynomial whose terms have a common factor, we factorize each term where each factor is that common factor. This is opposite of the product of a monomial and a polynomial where we used the distribution rule. \\

{\bf Example:} Multiply $3(x + 2y -z)$.\\
\vspace{2 cm}

{\bf Example:} Factorize $3x + 6y -3z $.\\
\vspace{2 cm}

{\bf Example:} Factorize $10y + 15 $.\\
\vspace{2 cm}

{\bf Example:} Factorize $8a -12$.\\
\vspace{2 cm}

{\bf Example:} Factorize $24 x^5 + 30 x^2$.\\
\vspace{2 cm}

{\bf Example:} Factorize $12 x^5 - 15 x^4 + 27 x^3$.\\
\vspace{2 cm}

{\bf Example:} Factorize $8r^3s^2 + 16 r s^3$.\\
\vspace{2 cm}

\subsection{Factoring by Grouping}

If not every term has a common factor, you can group the ones that have a common term and factor, and then see if there is a common factor between the groups. \\

First solve the example below to see that common factors do not have be a monomial, they can have more than one term.\\

{\bf Example:} Factorize $x^2(x+1) + 2(x+1)$.\\

Note that the binomial $x+1$ is a factor of both $x^2(x+1)$ and $2(x+1)$.\\
\vspace{2 cm}

Now we can apply the idea of grouping:\\

{\bf Example:} Factorize $5x^3 - x^2 + 15x -3$.\\

Hint: Group the first two and the last two terms together.\\
\vspace{3 cm}

{\bf Example:} Factorize $2x^3 + 8x^2 + x +4$.\\
\vspace{3 cm}

{\bf Example:} Factorize $8x^4 - 28x^3 + 6x - 21$.\\
\vspace{3 cm}

You can still use the evaluation method to check if you did the factorization right.\\

{\bf Example:} Check the factorization of $8x^4 - 28x^3 + 6x - 21$ above by evaluating at $x = 1$.\\
\vspace{3 cm}


 
\subsection{Factoring Trinomials}

This is opposite of the FOIL method we learned in Chapter 4. \\

{\bf Example:} Multiply $(x+2)(x+5)$.\\
\vspace{2 cm}

{\bf Example:} Multiply $(x-3)(x+7)$.\\
\vspace{2 cm}

{\bf Observations:}\\

1) The product of two binomial is a trinomial.\\

2) The coefficient of x is the sum of the constant terms of the binomials.\\

3) The constant coefficient of the trinomail is the product of the constant terms of the binomials.\\

Let's apply the observation to the question below:\\

{\bf Example:} Factorize $x^2 + 7x +10$.\\

1) The product of two binomial is a trinomial: $x^2 + 7x +10 = (x + p) (x + q)$ for some numbers p and q. \\

2) The coefficient of x is the sum of the constant terms of the binomials: $ p + q = 7$.\\

3) The constant coefficient of the trinomail is the product of the constant terms of the binomials: $ p \cdot q = 10$.\\

Can you guess now what $p$ and $q$ are? Find two numbers so that when you multiply them you get 10, and when you add them you get 7: .................\\

So the answer is $x^2 + 7x +10 = (x + .....) (x + .....)$\\

\newpage
{\bf Example:} Factorize $x^2 - 7x +10 $.\\
\vspace{4 cm}

{\bf Example:} Factorize $x^2 + 4x -21 $.\\
\vspace{4 cm}

{\bf Example:} Factorize $y^2 - 8y +12$.\\
\vspace{4 cm}

{\bf Example:} Factorize $a^2 + 4ab -21b^2 $.\\
\vspace{4 cm}

{\bf Example:} Factorize $x^2 -x +5$.\\
\vspace{4 cm}

{\bf What if there is a number in front of $x^2$?}\\

{\bf Example:} Multiply $(2x+5)(3x+4)$.\\
\vspace{2 cm}



{\bf Observations:}\\

1) The product of two binomial is a trinomial.\\

2) The coefficient of $x^2$ of the trinomial is the product of the coefficient of first terms of the binomials.\\

3) The constant coefficient of the trinomail is the product of the constant terms of the binomials.\\

4) The sum of the inner and outer products of the binomials is the term with degree one of the trinomial.\\ 

Let's apply the observation to the question below:\\

{\bf Example:} Factorize $3x^2 - 10x -8 $.\\

1) The product of two binomial is a trinomial: $3x^2 - 10x -8  = (...x + p) (...x + q)$ for some numbers p and q. \\

2)  The coefficient of $x^2$ of the trinomial is the product of the coefficient of first terms of the binomials: $3x^2 - 10x -8  = (3x + p) (x + q)$.\\

3) The constant coefficient of the trinomail is the product of the constant terms of the binomials: $ p \cdot q = -8 $.\\

Can you guess what $p$ and $q$ should be? Find two numbers so that when you multiply them you get -8: -1, 8 or 1,-8 or -2, 4 or ................................................\\

4) The sum of the inner and outer products of the binomials is the term with degree one of the trinomial: \\

Trial 1: $(3x -1)(x+8) = 3x^2 + 23x - 8$  \hspace{2cm} Wrong choice since $23 \neq -10$\\

Trial 2: $(3x +1)(x-8) = $...............................................................................\\

Trial 3: ...............................................................................\\

Trial 4: ...............................................................................\\

Trial 5: ...............................................................................\\

Trial 6: ...............................................................................\\

Trial 7: ...............................................................................\\

Trial 8: ...............................................................................\\

So the answer is $3x^2 -10x -8 = (x + .....) (x + .....)$\\

{\bf Hint:} If your choice let you a wrong answer just by a sign, change the signs of the numbers only.\\

{\bf Example:} Factorize $2x^2 +7x -4 $.\\
\vspace{4 cm}

{\bf Example:} Factorize $4x^2 +12x +5 $.\\
\vspace{4 cm}

{\bf Example:} Factorize $6y^2 - 10y -4 $.\\
\vspace{4 cm}

{\bf Example:} Factorize $-2a^2 +15 +a $.\\
\vspace{4 cm}

{\bf Example:} Factorize $18x^3 +21x^2 -9x$.\\
\vspace{4 cm}

{\bf Example:} Factorize $2a^2 - 5ab + 2b^2$.\\
\vspace{3 cm}



\subsection{Factoring polynomials that fit into a formula}

{\bf Review:} $(A-B)(A+B) = A^2 - B^2$\\

$(A-B)^2 = A^2 -2AB + B^2$\\

$(A+B)^2 = A^2 +2AB + B^2$\\


{\bf Formulas to Factorize:} $A^2 - B^2 =(A-B)(A+B) $\\

$A^2 -2AB + B^2 = (A-B)^2  $\\

$A^2 +2AB + B^2 = (A+B)^2  $\\

{\bf Study Skill:} You need to be able to recognize these forms to be able to factorize by using the formula!\\

{\bf Example:} Factorize $x^2 + 6x + 9 $.\\
\vspace{2 cm}

{\bf Example:} Factorize $t^2 -8t - 9$.\\
\vspace{2 cm}

{\bf Example:} Factorize $16x^2 + 49 -56x$.\\
\vspace{2 cm}

{\bf Example:} Factorize $9x^2 - 64$.\\
\vspace{2 cm}

{\bf Example:} Factorize $25 - t^3$.\\
\vspace{2 cm}

{\bf Example:} Factorize $-4x^{10} + 36$.\\
\vspace{2 cm}

\subsection{More Formulas for Sums and Differences of Cubes}

{\bf Formulas to Factorize:} $A^3 - B^3 =(A-B)(A^2 + AB + B^2) $\\

$A^3 + B^3 = (A+B)(A^2 -AB + B^2) $\\

{\bf Study Skill:} You may need to memorize squares and cubes of some small numbers to be able to recognize these forms.\\

{\bf Example:} Factorize $x^3 + 27 $.\\
\vspace{2 cm}

{\bf Example:} Factorize $8y^3 - x^3$.\\
\vspace{2 cm}

{\bf Study Skill:} To factorize, follow these steps:\\

1) First look for any common factor and take all out.\\

2) Then look at the numbers and see if you can recognize one of the formulas:\\

a) Two terms: $A^2 - B^2$ or  $A^3 - B^3$ or $A^3 + B^3$\\

b) Three terms: $(A+B)^2$ or $(A-B)^2$ or inverse of FOIL.\\

c) More terms: try grouping!\\

3) Make sure everything is factored completely.\\

4) Check answer by evaluating. \\

{\bf Example:} Factorize $5x^4 - 80$.\\
\vspace{3 cm}

{\bf Example:} Factorize $2x^3 + 10x^2 + x + 5$.\\
\vspace{3 cm}

{\bf Example:} Factorize $-x^5 + 2x^4 + 35x^3$.\\
\vspace{3 cm}

{\bf Example:} Factorize $x^2 -20x + 100 $.\\
\vspace{2 cm}

{\bf Example:} Factorize $x^2y^2 + 7xy +12$.\\
\vspace{3 cm}

{\bf Example:} Factorize $x^4 - 16y^4$.\\
\vspace{3 cm}

\subsection{Combining two topics: Factorization and Solving Equations}

{\bf Quadratic Equation:} $ax^2 + bx + c = 0$\\

{\bf Principle of zero:} If product of two factors is zero, then at least one of the factors has to be zero!\\

{\bf Example:} If $5 \cdot x = 0$, then what is x?\\

{\bf Example:} If $ x \cdot y = 0$, then what can you say about x and y?\\

So by using the principle of zero, we can solve equations by simply factorizing and setting each factor equal to zero and solving for x.\\

Make sure you have all the terms on one side of the equation and have only zero on the other side of the equation.\\

{\bf Example:} Solve $x^2 -8x +15 = 0$.\\
\vspace{3 cm}

{\bf Example:} Solve $6t^2 + t = 5$.\\
\vspace{3 cm}

{\bf Example:} Solve $4x + 6x^2 = 0$.\\
\vspace{3 cm}

{\bf Example:} Solve $25a^2 = 16$.\\
\vspace{3 cm}

{\bf Example:} Solve $m(m+5) = 14$.\\
\vspace{3 cm}

\subsection{Algebraic Applications}
{\bf Example:} Simplify: $\displaystyle \frac{5x+5}{15x^2+5x}$.\\

\vspace{3 cm}

{\bf Example:} Simplify: $\displaystyle \frac{4x-8}{3x-6}$.\\

\vspace{3 cm}

{\bf Example:} Simplify: $\displaystyle \frac{x^2+2x-8}{x^2-4}$.\\

\vspace{3 cm}

{\bf Example:} Multiply and simplify: $\displaystyle \frac{3b^5}{14} \cdot \frac{7}{3b^2}$.\\

\vspace{3 cm}

{\bf Example:} Multiply and simplify: $\displaystyle \frac{x^2+4x+3}{x^2-4} \cdot \frac{x-2}{x+1}$.\\

\vspace{4 cm}

{\bf Example:} Multiply and simplify: $\displaystyle \frac{x}{9} \div \frac{x^2}{18}$.\\

\vspace{3 cm}

{\bf Example:} Multiply and simplify: $\displaystyle \frac{2y+8}{y} - \frac{8}{y}$.\\

\vspace{3 cm}

{\bf Example:} Multiply and simplify: $\displaystyle \frac{4x}{x-7} + \frac{3x-49}{x-7}$.\\

\vspace{3 cm}

{\bf Example:} Multiply and simplify: $\displaystyle \frac{2}{3x^2} - \frac{1}{4x}$.\\

\vspace{3 cm}


\subsection{Work Problems}

Use the methods we learned before to translate the wording into math equations. Proceed as in the previous section to solve for the unknowns.\\


{\bf Example:} The product of two consecutive negative integers is 1122. What are the numbers?\\
\vspace{3 cm}

{\bf Example:} A certain number added to its square is 30.  Find the number.\\
\vspace{3 cm}

{\bf Example:} The ages of three family children can be expressed as consecutive integers.  The square of the age of the youngest child is 4 more than eight times the age of the oldest child.  Find the ages of the three children.\\
\vspace{3 cm}

{\bf Example:} A square and rectangle have the same area.  The length of the 
rectangle is five inches more than twice the length of the side of the 
square.  The width of the rectangle is 6 inches less than the side of 
the square.  Find the length of the side of the square.\\
\vspace{5 cm}

{\bf Example:} If the measure of one side of a square is increased by 2 centimeters and the measure of the adjacent side is decreased by 2 centimeters, the area of the resulting rectangle is 32 square centimeters.  Find the measure of one side of the square.\\

\begin{problem}
\begin{sagesilent}
p=-x^2+3*x+9
q=-2
\end{sagesilent}
 Evaluate $\sage{p}$ for $x = \sage{q}$. $Value=\answer{\sage{p(q)}}$
\end{problem}


\begin{problem}
\begin{sagesilent}
t=var('t')
p=t^2-5*t^5+9-4*t^3
q=0
\end{sagesilent}
 Evaluate $\sage{p}$ for $t= \sage{q}$. $Value=\answer{\sage{p(q)}}$
\end{problem}

\begin{problem}
\begin{sagesilent}
t=var('t')
p=t^2-5*t^5+9-4*t^3
q=1
\end{sagesilent}
 Evaluate $\sage{p}$ for $t= \sage{q}$. $Value=\answer{\sage{p(q)}}$
\end{problem}

\begin{problem}
\begin{sagesilent}
t=var('t')
p=t^2-5*t^5+9-4*t^3
q=-1
\end{sagesilent}
 Evaluate $\sage{p}$ for $t= \sage{q}$. $Value=\answer{\sage{p(q)}}$
\end{problem}


\begin{problem}
\begin{sagesilent}
p=3+5*x
q=8*x
\end{sagesilent}
True or False: $\sage{p}=\sage{q}$
\begin{multipleChoice}
    \choice{True}
    \choice[correct]{False}
\end{multipleChoice}
\end{problem}



\begin{problem}
\begin{sagesilent}
p=3*x^2+5*x
q=8*x^3
\end{sagesilent}
True or False: $\sage{p}=\sage{q}$
\begin{multipleChoice}
    \choice{True}
    \choice[correct]{False}
\end{multipleChoice}
\end{problem}



\begin{problem}
\begin{sagesilent}
p=3*x+5*x
q=8*x
\end{sagesilent}
True or False: $3x+5x=\sage{q}$
\begin{multipleChoice}
    \choice[correct]{True}
    \choice{False}
\end{multipleChoice}
\end{problem}

\begin{problem}
Combine the like terms:

\begin{enumerate}
    \item $7x^3 - 5x^2 + 9x^3 + x^2 = \answer{\sage{7*x^3 - 5*x^2 + 9*x^3 + x^2}} $
    \item $\displaystyle \frac{2}{3} x^4 - x^3 - \frac{1}{6} x^4 + \frac{2}{5} x^3 - \frac{3}{10} x^3 = \answer{\sage{2/3*x^4 - x^3 - 1/6*x^4 + 2/5*x^3 - 3/10*x^3}} $
\end{enumerate}
\end{problem}

\begin{problem}
Add or subtract:

\begin{enumerate}
    \item $(-5x^3 + 6x -1) + (4x^3 + 3x^2 + 2)
    =\answer{0}\cdot x^4+\answer{-1}\cdot x^3+\answer{3}\cdot x^2+\answer{6}\cdot x+\answer{1}
    \\=\answer{\sage{(-5*x^3 + 6*x -1) + (4*x^3 + 3*x^2 + 2)}}
    $
    \item $\displaystyle( \frac{2}{3} x^4 + 3x^2 - 7x + \frac{1}{2}) + ( - \frac{1}{3} x^4 + 5x^3 - 3x^2 + 3x - \frac{1}{2}) 
    =\answer{1/3}\cdot x^4+\answer{5}\cdot x^3+\answer{0}\cdot x^2+\answer{-4}\cdot x+\answer{0}
    \\= \answer{\sage{( 2/3*x^4 + 3*x^2 - 7*x + 1/2) + ( -1/3*x^4 + 5*x^3 - 3*x^2 + 3*x - 1/2) }} $
    \item $(9x^5 + x^3 - 2x^2 + 4) - (-2x^5 + x^4 -4x^3 -3x^2) 
        =\answer{0}\cdot x^6+\answer{11}\cdot x^5+\answer{-1}\cdot x^4+\answer{5}\cdot x^3+\answer{1}\cdot x^2+\answer{0}\cdot x+\answer{4}
        \\= \answer{\sage{(9*x^5 + x^3 - 2*x^2 + 4) - (-2*x^5 + x^4 -4*x^3 -3*x^2)}}$
    \item $(7x^5 + x^3 - 9x) - (3x^5 - 4x^3 + 5) 
    =\answer{4}\cdot x^5+\answer{0}\cdot x^4+\answer{5}\cdot x^3+\answer{0}\cdot x^2+\answer{-9}\cdot x+\answer{-5}
    \\= \answer{\sage{(7*x^5 + x^3 - 9*x) - (3*x^5 - 4*x^3 + 5)}}
    $
\end{enumerate}
\end{problem}

\begin{problem}
$(5x)\cdot (6x) = \answer{30x^2}$
\end{problem}

\begin{problem}
$(-x)(3x) = \answer{-3x^2} $
\end{problem}

\begin{problem}
$(7x^5)(-3x^4) = \answer{-21x^9}$
\end{problem}

\begin{problem}
$ x\cdot (x+3) = \answer{x^2+3x}$
\end{problem}

\begin{problem}
$5x(2x^2 - 3x +4) = \answer{10x^3-15x^2+20x}$
\end{problem}

\begin{problem}
$2 x^2  (x^3 - 7x^2 + 10x -4) =  \answer{2x^5-14x^4+20x^3-8x^2}$
\end{problem}

\begin{problem}
$ (x+5) \cdot (x+3) = \answer{x^2+8x+15}$
\end{problem}

\begin{problem}
$(4x-3)(x-2) = \answer{4x^2+6-11x}$
\end{problem}

\begin{problem}
$(x^2 + 2x -3) (x + 4) =  \answer{\sage{(x^2 + 2*x -3)*(x + 4)}}$
\end{problem}

\begin{problem}
$(x +8)(x^2+5)=\answer{x^3+8x^2+5x+40}$
\end{problem}

\begin{problem}
$(x + 8)(x-8)=\answer{x^2-64}$
\end{problem}

\begin{problem}
$(2x^3 + 5)(2x^3 - 5)=\answer{4x^6-25}$
\end{problem}

\begin{problem}
$(x+7)^2=\answer{x^2+14x+49}$
\end{problem}

\begin{problem}
$(t-5)^2 = \answer{t^2-10t+25}$
\end{problem}

\begin{problem}
$(3a + 4)^2 = \answer{9a^2+24a+16}$
\end{problem}

\begin{problem}
True or False: $ (5 + x)^2 = 5^2 + x^2 $
\begin{multipleChoice}
    \choice{True}
    \choice[correct]{False}
\end{multipleChoice}
\end{problem}

\begin{problem}
\begin{sagesilent}
x=var('x')
y=var('y')
p=4 + 3*x + x*y^2 + 8*x^3*y^3
qx=-2
qy=5
\end{sagesilent}
Evaluate the polynomial $\sage{p}$ for $x=\sage{qx}$ and $y=\sage{qy}$.
$$Answer=\answer{\sage{p(qx,qy)}}$$
Degree of the polynomial is $\answer{6}$.
\end{problem}


\begin{problem}
\begin{sagesilent}
x=var('x')
y=var('y')
p=-4 -30*x + 3*y^2 -10*x^2*y^3
qx=1
qy=-1
\end{sagesilent}
Evaluate the polynomial $\sage{p}$ for $x=\sage{qx}$ and $y=\sage{qy}$.
$$Answer=\answer{\sage{p(qx,qy)}}$$
Degree of the polynomial is $\answer{5}$
\end{problem}


\begin{problem}
Combine the like terms: $9x^2y + 3xy^2 - 5x^2y - xy^2=\answer{\sage{9*x^2*y + 3*x*y^2 - 5*x^2*y - x*y^2}}$
\end{problem}
\begin{problem}
\begin{sagesilent}
a=var('a')
b=var('b')
\end{sagesilent}
Add: $(5ab^2 - 4a^2b + 5a^3 + 2) + (3ab^2 - 2a^2b + 3a^3 - 5) =\answer{\sage{(5*a*b^2 - 4*a^2*b + 5*a^3 + 2) + (3*a*b^2 - 2*a^2*b + 3*a^3 - 5) }}$
\end{problem}



\begin{problem}
Subtract: $(4x^2y + x^3y^2 + 3x^2y^3 + 6y)- (4x^2y - 6x^3y^2 + x^2y^2 - 5y) 
=\answer{\sage{(4*x^2*y + x^3*y^2 + 3*x^2*y^3 + 6*y)- (4*x^2*y - 6*x^3*y^2 + x^2*y^2 - 5*y) }}$
\end{problem}

\begin{problem}
\begin{sagesilent}
x=var('x')
y=var('y')
\end{sagesilent}

Multiply: $(3x^2y - 2xy + 3y)(xy + 2y)=\answer{\sage{(3*x^2*y - 2*x*y + 3*y)*(x*y + 2*y)}} $
\end{problem}

\begin{problem}
\begin{sagesilent}
x=var('x')
y=var('y')
\end{sagesilent}

Multiply: $(3x^2y +2y)(3x^2y - 2y)=\answer{\sage{(3*x^2*y +2*y)*(3*x^2*y - 2*y)}}$
\end{problem}




\begin{problem}
\begin{sagesilent}
x=var('x')
y=var('y')
\end{sagesilent}
Multiply: $(3x +2y)^2=\answer{9x^2+4y^2+12xy}$
\end{problem}

\begin{problem}
\begin{sagesilent}
x=var('x')
y=var('y')
\end{sagesilent}
Multiply: $(2x + 3 - 2y)(2x + 3 + 2y) =\answer{\sage{(2*x + 3 - 2*y)*(2*x + 3 + 2*y)}}$
\end{problem}

\begin{problem}
$\displaystyle \frac{15 x^{10}}{3 x^4} =\answer{5x^6}$
\end{problem}

\begin{problem}
$$\displaystyle \frac{86x^5}{2x^3} = \answer{43x^2}$$
\end{problem}

\begin{problem}
$\displaystyle \frac{80 x^{5} + 6 x^7}{2 x^3} =\answer{40x^2+3x^4}$
\end{problem}

\begin{problem}
$\displaystyle \frac{x^{4} + 15 x^3 - 6x^2}{3x} =\answer{x^3/3+5x^2-2x}$
\end{problem}

\begin{problem}
$\displaystyle (10a^5b^4 - 2a^3b^2 + 6 a^2b) \div (-2a^2b) =\answer{-5a^3b^3+ab-3}$
\end{problem}

\begin{problem}
Write the division as an integer plus a fraction between 0 and 1: 
$$85,582 \div 24 =\answer{\sage{floor(85582/24)}}+\frac{\answer{\sage{85582-floor(85582/24)*24}}}{24}$$
\end{problem}


\begin{problem}
\begin{sagesilent}
p(x)=(2*x^4 - 9*x^3 + 21*x^2 - 26*x + 12)
q(x)=( 2*x-3 )
s=p.maxima_methods().divide(q)
\end{sagesilent}
Write the division as a polynomial plus a fraction whose top polynomial has a degree less than $\sage{q(x)}$: 
$$(\sage{p(x)})\div (\sage{q(x)}) =\answer{\sage{s[0]}} +\frac{\answer{\sage{s[1]}}}{\sage{q(x)}}
$$
\end{problem}


\begin{problem}
\begin{sagesilent}
p(x)=(x^3+1)
q(x)=( x+1 )
s=p.maxima_methods().divide(q)
\end{sagesilent}
Write the division as a polynomial plus a fraction whose top polynomial has a degree less than $\sage{q(x)}$: 
$$(\sage{p(x)})\div (\sage{q(x)}) =\answer{\sage{s[0]}} +\frac{\answer{\sage{s[1]}}}{\sage{q(x)}}$$
\end{problem}


\begin{problem}
\begin{sagesilent}
p(x)=(2*x^2 + 5*x - 1)
q(x)=( 2*x-1 )
s=p.maxima_methods().divide(q)
\end{sagesilent}
Write the division as a polynomial plus a fraction whose top polynomial has a degree less than $\sage{q(x)}$: 
$$(\sage{p(x)})\div (\sage{q(x)}) =\answer{\sage{s[0]}} +\frac{\answer{\sage{s[1]}}}{\sage{q(x)}}$$
\end{problem}

\begin{problem}
\begin{sagesilent}
p(x)=(3*x^4 - 2*x^3 -6*x + 1)
q(x)=( 2*x^2-1 )
s=p.maxima_methods().divide(q)
\end{sagesilent}
Write the division as a polynomial plus a fraction whose top polynomial has a degree less than $\sage{q(x)}$: 
$$(\sage{p(x)})\div (\sage{q(x)}) =\answer{\sage{s[0]}} +\frac{\answer{\sage{s[1]}}}{\sage{q(x)}}$$
\end{problem}

\end{document}


\begin{javascript}
  function isPositive(number) {
    return number &gt; 0;
  };
 
  function sameParity(a,b) {
    return (a-b)%2 == 0;
  };
 
  caseInsensitive = function(a,b) {
    return a.toLowerCase() == b.toLowerCase();
  };
 
  sameDerivative = function(a,b) {
    return a.derivative('x').equals( b.derivative('x') );
  };
\end{javascript}
