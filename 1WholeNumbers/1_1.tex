\documentclass{ximera}
%% You can put user macros here
%% However, you cannot make new environments

\usepackage[letterpaper, total={6in, 8in}]{geometry}
\graphicspath{{./}{firstExample/}{secondExample/}}

\usepackage{tikz}
\usepackage{tkz-euclide}
\usetkzobj{all}
\usetikzlibrary{arrows}
\tikzstyle geometryDiagrams=[ultra thick,color=blue!50!black]

%\usepackage[dvipsnames]{xcolor}

\title{Arithmetic Laws and Order of Operations \\ Homework 2}
\author{Oguz Kurt}
\begin{abstract}
    \empty
\end{abstract}

%\outcome{Identify, distinguish natural, whole numbers and integers.}
%\outcome{Use natural, whole numbers and integers in operations.}
%\outcome{Learn the impact of different operations on natural, whole numbers and integers.}

\begin{document}
\maketitle


%\begin{shuffle}[1]
\section*{Arithmetic Laws}
\begin{problem} Answer the following problems.

$38 - 62 =\answer{\sage{38-62}}$.

$38 - (-62) =\answer{\sage{38-(-62)}}$.

$-38 - 62 =\answer{\sage{-38-(62)}}$.

$-38 - (-62) =\answer{\sage{-38-(-62)}}$.

$12\times (2)=\answer{\sage{12*2}}$.

$12\times (-2)=\answer{\sage{12*-2}}$.

$(-12)\times 2=\answer{\sage{-12*2}}$.

$(-12)\times (-2)=\answer{\sage{-12*-2}}$.

$-(-10)=\answer{\sage{--10}}$

$-(-(-10))=\answer{\sage{---10}}$

$-(-(-(-10)))=\answer{\sage{----10}}$

$24\div 8=\answer{\sage{24/8}}$.

$24\div (-8)=\answer{\sage{24/-8}}$.

$-24\div 8=\answer{\sage{-24/8}}$.

$-24\div (-8)=\answer{\sage{-24/-8}}$.

$3^3=\answer{\sage{3^3}}$.

$(-3)^3=\answer{\sage{(-3)^3}}$.

$(-3)^4=\answer{\sage{(-3)^4}}$.

$-3^4=\answer{\sage{-3^4}}$.

\end{problem}

\begin{problem}
Which arithmetic law is used in the following operation:
$$36-15+0=36-15$$

\begin{multipleChoice}
    \choice{Commutative Law}
    \choice{Associative Law}
    \choice[correct]{Identity Law}
    \choice{Distributive Law}
\end{multipleChoice}
\end{problem}



\begin{problem}
Which arithmetic law is used in the following operation:
$$36-15+2=21+2$$

\begin{multipleChoice}
    \choice{Commutative Law}
    \choice[correct]{Associative Law}
    \choice{Identity Law}
    \choice{Distributive Law}
\end{multipleChoice}
\end{problem}


\begin{problem}
Which arithmetic law is used in the following operation:
$$36-15+2=36+2-15$$

\begin{multipleChoice}
    \choice[correct]{Commutative Law}
    \choice{Associative Law}
    \choice{Identity Law}
    \choice{Distributive Law}
\end{multipleChoice}
\end{problem}

\begin{problem}
Which arithmetic law is used in the following operation:
$$3\cdot (6-15)=3\cdot 6+ 3\cdot (-15)$$

\begin{multipleChoice}
    \choice{Commutative Law}
    \choice{Associative Law}
    \choice{Identity Law}
    \choice[correct]{Distributive Law}
\end{multipleChoice}
\end{problem}


\begin{problem}
    $(-2)(31-5)=(\answer{-2})\cdot 31+(-2)\cdot (\answer{-5})$
\begin{hint}
Please, use Distributive Law
\end{hint}
\end{problem}

\section*{Order of Operations}


\begin{problem}
\begin{sagesilent}
a=randint(0,10)
b=randint(0,20)
c=randint(0,10)
d=randint(0,20)
\end{sagesilent}
$\sage{a}-\sage{b}=\answer{\sage{a-b}} $
\end{problem}

\begin{problem}
\begin{sagesilent}
a=randint(0,10)
b=randint(0,20)
c=randint(0,10)
d=randint(0,20)
\end{sagesilent}
$\sage{a}-\sage{b}=\answer{\sage{a-b}} $
\end{problem}


\begin{problem}
\begin{sagesilent}
a=randint(0,10)
b=randint(0,20)
c=randint(0,10)
d=randint(0,20)
g=randint(1,5)
\end{sagesilent}
$ \sage{a} + (\sage{b} \times \sage{c}^{\sage{g}} + \sage{d}) =\answer{\sage{a+(b*c^g+d)}} $
\end{problem}


\begin{problem}
\begin{sagesilent}
a=randint(0,10)
b=randint(0,20)
c=randint(0,10)
d=randint(0,20)
g=randint(1,5)
\end{sagesilent}
$ \sage{a} + (\sage{b} \times \sage{c}^{\sage{g}} + \sage{d}) =\answer{\sage{a+(b*c^g+d)}} $
\end{problem}

\begin{problem}
\begin{sagesilent}
a=randint(0,10)
b=randint(0,20)
c=randint(0,10)
d=randint(0,20)
g=randint(1,5)
\end{sagesilent}
$ \sage{a} -(\sage{b} \times \sage{c}^{\sage{g}} + \sage{d}) =\answer{\sage{a-(b*c^g+d)}} $
\end{problem}

\begin{problem}
\begin{sagesilent}
a=randint(0,10)
b=randint(0,20)
c=randint(0,10)
d=randint(0,20)
g=randint(1,5)
\end{sagesilent}
$ \sage{a} - (\sage{b} \times \sage{c}^{\sage{g}} + \sage{d}) =\answer{\sage{a-(b*c^g+d)}} $
\end{problem}


\begin{problem}
\begin{sagesilent}
a=randint(0,10)
b=randint(0,20)
c=randint(0,10)
d=randint(0,20)
g=randint(1,5)
\end{sagesilent}
$ \sage{a} + (\sage{b} \times \sage{c}^{\sage{g}} - \sage{d}) =\answer{\sage{a+(b*c^g-d)}} $
\end{problem}

\begin{problem}
\begin{sagesilent}
a=randint(0,10)
b=randint(0,20)
c=randint(0,10)
d=randint(0,20)
g=randint(1,5)
\end{sagesilent}
$ \sage{a} - (-\sage{b} \times \sage{c}^{\sage{g}} + \sage{d}) =\answer{\sage{a-(-b*c^g+d)}} $
\end{problem}


\begin{problem}
$368\times 5 \div 10 \times 2\div 8=\answer{\sage{368*5/10*2/8}}$.
\end{problem}


\begin{problem}
$2\cdot(53-3^4)-(2\cdot 18\div 12+2)=\answer{\sage{2*(53-3^4)-(2*18/12+2)}}$.

\end{problem}



\begin{problem}
$ 7 + (6 \times 5^2 + 3) =\answer{\sage{7+(6*5^2+3)}} $
\end{problem}

\section*{Algebraic Expressions}

\begin{problem}
\begin{sagesilent}
a=randint(0,10)
b=randint(0,100)
c=randint(0,100)
\end{sagesilent}
Distribute $\sage{a}\cdot (\sage{b}x+\sage{c})=\answer{\sage{a*(b*x+c)}}$
\end{problem}

\begin{problem}
\begin{sagesilent}
a=randint(0,10)
b=randint(0,100)
c=randint(0,100)
\end{sagesilent}
Distribute $-\sage{a}\cdot (\sage{b}x+\sage{c})=\answer{\sage{-a*(b*x+c)}}$
\end{problem}

\begin{problem}
\begin{sagesilent}
a=randint(0,10)
b=randint(0,100)
c=randint(0,100)
\end{sagesilent}
Distribute $-\sage{a}\cdot (-\sage{b}x+\sage{c})=\answer{\sage{-a*(-b*x+c)}}$
\end{problem}

\begin{problem}
\begin{sagesilent}
a=randint(0,10)
b=randint(0,100)
c=randint(0,100)
\end{sagesilent}
Distribute $-\sage{a}\cdot (\sage{b}x-\sage{c})=\answer{\sage{-a*(b*x-c)}}$
\end{problem}

\begin{problem}
\begin{sagesilent}
a=randint(0,10)
b=randint(0,100)
c=randint(0,100)
\end{sagesilent}
Distribute $-\sage{a}\cdot (-\sage{b}x-\sage{c})=\answer{\sage{-a*(-b*x-c)}}$
\end{problem}


\begin{problem}
\begin{sagesilent}
y=var('y')
a=randint(0,100)
b=randint(0,100)
c=randint(0,100)
d=randint(0,100)
\end{sagesilent}
Simplify $\sage{a}x+\sage{b}y-\sage{c}x-\sage{d}y=\answer{\sage{a*x+b*y-c*x-d*y}}$

\begin{hint}
\item[(1)] Use Commutativity to bring the like terms together, 
\item[(3)] Next, use Associativity to put the like terms in the same paranthesis 
\item[(3)] Finally, use Distributivity to add the coefficients of the like terms.
\end{hint}
\end{problem}

\begin{problem}
\begin{sagesilent}
y=var('y')
a=randint(0,100)
b=randint(0,100)
c=randint(0,100)
d=randint(0,100)
\end{sagesilent}
Simplify $\sage{a}x-\sage{b}y-\sage{c}x-\sage{d}y=\answer{\sage{a*x-b*y-c*x-d*y}}$

\begin{hint}
\item[(1)] Use Commutativity to bring the like terms together, 
\item[(3)] Next, use Associativity to put the like terms in the same paranthesis 
\item[(3)] Finally, use Distributivity to add the coefficients of the like terms.
\end{hint}

\end{problem}


\begin{problem}
\begin{sagesilent}
y=var('y')
a=randint(0,100)
b=randint(0,100)
c=randint(0,100)
d=randint(0,100)
\end{sagesilent}
Simplify $\sage{a}x^2+\sage{b}x-\sage{c}x-\sage{d}x^2=\answer{\sage{a*x^2+b*x-c*x-d*x^2}}$

\begin{hint}
\item[(1)] Use Commutativity to bring the like terms together, 
\item[(3)] Next, use Associativity to put the like terms in the same paranthesis 
\item[(3)] Finally, use Distributivity to add the coefficients of the like terms.
\end{hint}

\end{problem}


\begin{problem}
\begin{sagesilent}
a=randint(0,100)
b=randint(0,100)
c=randint(0,100)
d=randint(0,10)
\end{sagesilent}
Simplify $-(\sage{a}x^2+\sage{b})=\answer{\sage{-(a*x^2+b)}}$

\begin{hint}
Use Distributivity to get rid of the paranthesis
\end{hint}

\end{problem}


\begin{problem}
\begin{sagesilent}
a=randint(0,100)
b=randint(0,100)
c=randint(0,10)
d=randint(0,100)
\end{sagesilent}
Simplify $-\sage{c}(\sage{a}x^2-\sage{b})=\answer{\sage{-c*(a*x^2-b)}}$

\begin{hint}
Use Distributivity to get rid of the paranthesis
\end{hint}
\end{problem}


\begin{problem}
\begin{sagesilent}
a=randint(0,100)
b=randint(0,100)
c=randint(0,10)
d=randint(0,100)
\end{sagesilent}
Simplify $x(\sage{a}x^2-\sage{b})=\answer{\sage{expand(x*(a*x^2-b))}}$

\begin{hint}
Use Distributivity to get rid of the paranthesis
\end{hint}
\end{problem}

\begin{problem}
\begin{sagesilent}
a=randint(0,100)
b=randint(0,100)
c=randint(0,10)
d=randint(0,100)
\end{sagesilent}
Simplify $-x(\sage{a}x^2-\sage{b})=\answer{\sage{expand(-x*(a*x^2-b))}}$

\begin{hint}
Use Distributivity to get rid of the paranthesis
\end{hint}
\end{problem}


\begin{problem}
\begin{sagesilent}
a=randint(1,5)
b=randint(1,5)
c=randint(0,10)
d=randint(0,100)
\end{sagesilent}
Simplify $(\sage{a}x)^{\sage{b}}=\answer{\sage{expand((a*x)^b)}}$

\begin{hint}
Use the definition of the exponent $a^k=a\cdot a\cdot \ldots \cdot a$ (multiply by itself k times.)
\end{hint}
\end{problem}


\begin{problem}
\begin{sagesilent}
a=randint(1,5)
b=randint(1,5)
c=randint(0,10)
d=randint(0,100)
\end{sagesilent}
Simplify $-(-\sage{a}x)^{\sage{b}}=\answer{\sage{expand(-(-a*x)^b)}}$

\begin{hint}
Use the definition of the exponent $a^k=a\cdot a\cdot \ldots \cdot a$ (multiply by itself k times.)
\end{hint}
\end{problem}


\begin{problem}
\begin{sagesilent}
a=randint(1,5)
b=randint(1,5)
c=randint(0,10)
d=randint(0,100)
\end{sagesilent}
Simplify $(-\sage{a}x)^{\sage{b}}=\answer{\sage{expand((-a*x)^b)}}$

\begin{hint}
Use the definition of the exponent $a^k=a\cdot a\cdot \ldots \cdot a$ (multiply by itself k times.)
\end{hint}
\end{problem}





\end{document}
