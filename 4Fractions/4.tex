\documentclass{ximera}
%% You can put user macros here
%% However, you cannot make new environments

\usepackage[letterpaper, total={6in, 8in}]{geometry}
\graphicspath{{./}{firstExample/}{secondExample/}}

\usepackage{tikz}
\usepackage{tkz-euclide}
\usetkzobj{all}

\tikzstyle geometryDiagrams=[ultra thick,color=blue!50!black]

\title{Fractions \\ Homework 4}
\author{Oguz Kurt}
\begin{abstract}
    Learn, identify, distinguish fractions and use them in calculations.
\end{abstract}

\outcome{Identify, distinguish natural, whole numbers and integers.}
\outcome{Use natural, whole numbers and integers in operations.}
\outcome{Learn the impact of different operations on natural, whole numbers and integers.}

\begin{document}
\maketitle


%\begin{shuffle}[1]

\begin{problem}
\begin{sagesilent}
a=5
b=8
c=18
d=8
\end{sagesilent}
\begin{prompt}
$$\frac{\sage{a}}{\sage{b}}+\frac{\sage{c}}{\sage{d}}=\answer{\sage{a/b+c/d}}$$
\end{prompt}
\end{problem}

\begin{problem}
\begin{sagesilent}
a=5
b=8
c=18
d=8
\end{sagesilent}
\begin{prompt}
$$\frac{\sage{a}}{\sage{b}}-\frac{\sage{c}}{\sage{d}}=\answer{\sage{a/b-c/d}}$$
\end{prompt}
\end{problem}

\begin{problem}
\begin{sagesilent}
a=5
b=8
c=18
d=8
\end{sagesilent}
\begin{prompt}
$$\frac{\sage{a}}{\sage{b}}\cdot\frac{\sage{c}}{\sage{d}}=\answer{\sage{a/b*c/d}}$$
\end{prompt}
\end{problem}

\begin{problem}
\begin{sagesilent}
a=5
b=8
c=18
d=8
\end{sagesilent}
\begin{prompt}
$$\frac{\sage{a}}{\sage{b}}\div\frac{\sage{c}}{\sage{d}}=\answer{\sage{(a/b)/(c/d)}}$$
\end{prompt}
\end{problem}

\begin{problem}
\begin{sagesilent}
a=5
b=8
c=18
d=8
\end{sagesilent}
\begin{prompt}
$$\frac{\sage{c}}{\sage{a}}=\sage{floor(c/a)}+\answer{\sage{c/a-floor(c/a)}}=\sage{ceil(c/a)}-\answer{\sage{ceil(c/a)-c/a}} $$
\end{prompt}
\end{problem}


\begin{problem}
\begin{sagesilent}
a=5
b=8
c=-18
d=8
\end{sagesilent}
\begin{prompt}
$$\frac{\sage{c}}{\sage{a}}=\sage{floor(c/a)}+\answer{\sage{c/a-floor(c/a)}}=\sage{ceil(c/a)}-\answer{\sage{ceil(c/a)-c/a}} $$
\end{prompt}
\end{problem}


\begin{problem}
\begin{sagesilent}
a=7
b=8
c=24
d=8
\end{sagesilent}
\begin{prompt}
$$\frac{\sage{c}}{\sage{a}}=\answer{\sage{floor(c/a)}}+{\sage{c/a-floor(c/a)}}=\answer{\sage{ceil(c/a)}}-{\sage{ceil(c/a)-c/a}} $$
\end{prompt}
\end{problem}

\begin{problem}
\begin{sagesilent}
a=6
b=8
c=-27
d=8
\end{sagesilent}
\begin{prompt}
$$\frac{\sage{c}}{\sage{a}}=\answer{\sage{floor(c/a)}}+{\sage{c/a-floor(c/a)}}=\answer{\sage{ceil(c/a)}}-{\sage{ceil(c/a)-c/a}} $$
\end{prompt}
\end{problem}

\begin{problem}
\begin{sagesilent}
a=11
b=8
c=80
d=8
\end{sagesilent}
Please, write the following fraction as an addition/subtraction of an integer and a fraction between 0 and 1.
\begin{prompt}
$$\frac{\sage{c}}{\sage{a}}=\answer{\sage{floor(c/a)}}+\answer{\sage{c/a-floor(c/a)}}=\answer{\sage{ceil(c/a)}}-\answer{\sage{ceil(c/a)-c/a}} $$
\end{prompt}
\end{problem}


\begin{problem}
\begin{sagesilent}
a=11
b=8
c=-90
d=8
\end{sagesilent}
Please, write the following fraction as an addition of an integer and a fraction between 0 and 1.
\begin{prompt}
$$\frac{\sage{c}}{\sage{a}}=\answer{\sage{floor(c/a)}}+\answer{\sage{c/a-floor(c/a)}}=\answer{\sage{ceil(c/a)}}-\answer{\sage{ceil(c/a)-c/a}} $$
\end{prompt}
\end{problem}


\begin{problem}
\begin{sagesilent}
a=5
b=4
c=18
d=12
\end{sagesilent}
Simplify the following expression. If the expression is negative, write the numerator as a negative number:
\begin{prompt}
$$\frac{\frac{\sage{a}}{\sage{b}}}{\frac{\sage{c}}{\sage{d}}}=\frac{\answer{\sage{numerator((a/b)/(c/d))}}}{\answer{\sage{denominator((a/b)/(c/d))}}}$$
\end{prompt}
\end{problem}

\begin{problem}
\begin{sagesilent}
a=52
b=14
c=8
d=13
\end{sagesilent}
Simplify the following expression. If the expression is negative, write the numerator as a negative number:
\begin{prompt}
$$\frac{{\sage{a}}}{\frac{\sage{c}}{\sage{d}}}=\frac{\answer{\sage{numerator((a)/(c/d))}}}{\answer{\sage{denominator((a)/(c/d))}}}$$
\end{prompt}
\end{problem}


\begin{problem}
\begin{sagesilent}
a=52
b=14
c=13
d=13
\end{sagesilent}
Simplify the following expression. If the expression is negative, write the numerator as a negative number:
\begin{prompt}
$$\frac{\frac{\sage{a}}{\sage{b}}}{{\sage{c}}}=\frac{\answer{\sage{numerator((a/b)/c)}}}{\answer{\sage{denominator((a/b)/(c))}}}$$
\end{prompt}
\end{problem}


\begin{problem}
\begin{sagesilent}
a=50
b=14
c=-180
d=49
\end{sagesilent}
Simplify the following expression.  If the expression is negative, write the numerator as a negative number:
\begin{prompt}
$$\frac{\frac{\sage{a}}{\sage{b}}}{\frac{\sage{c}}{\sage{d}}}=\frac{\answer{\sage{numerator((a/b)/(c/d))}}}{\answer{\sage{denominator((a/b)/(c/d))}}}$$
\end{prompt}
\end{problem}


\begin{problem}
\begin{sagesilent}
a=5
b=-18
c=-15
d=21
\end{sagesilent}
Simplify the following expression.  If the expression is negative, write the numerator as a negative number:
\begin{prompt}
$$\frac{\frac{\sage{a}}{\sage{b}}+\frac{\sage{c}}{\sage{d}}}{\frac{\sage{c}}{\sage{d}}}=\frac{\answer{\sage{numerator((a/b+c/d)/(c/d))}}}{\answer{\sage{denominator((a/b)/(c/d))}}}$$
\end{prompt}
\end{problem}


\begin{problem}
\begin{sagesilent}
a=var('a')
b=var('b')
c=-15
d=21
\end{sagesilent}
Simplify the following expression.  If the expression is negative, write the numerator as a negative number:
\begin{prompt}
$$\frac{\frac{\sage{a}}{\sage{b}}+\frac{\sage{b}}{\sage{a}}}{\frac{\sage{a}}{\sage{b}}}=\frac{\answer{\sage{numerator((a/b+b/a)/(a/b))}}}{\answer{\sage{denominator((a/b)/(c/d))}}}$$
\end{prompt}
\end{problem}

\begin{problem}
\begin{sagesilent}
a=6
b=10
c=15
d=25
\end{sagesilent}
Simplify the following expression.  If the expression is negative, write the numerator as a negative number:
\begin{prompt}
$$\frac{\sage{a}x+\sage{b}}{\sage{c}x+\sage{d}}=\frac{\answer{\sage{numerator((a*x+b)/(c*x+d))}}}{\answer{\sage{denominator((a*x+b)/(c*x+d))}}}$$
\end{prompt}
\end{problem}

\end{document}


























\begin{problem}
Please, write \textbf{NI} for \textbf{not integer} in the answer box if the answer is not an integer.

$28+35=\answer{\sage{28+35}}$
\end{problem}

\begin{problem}
Please, write \textbf{NI} for \textbf{not integer} in the answer box if the answer is not an integer.

$28-35=\answer{\sage{28-35}}$
\end{problem}



\begin{problem}
Please, write \textbf{NI} for \textbf{not integer} in the answer box if the answer is not an integer.

$22-22=\answer{\sage{22-22}}$
\end{problem}


\begin{problem}
Please, write \textbf{NI} for \textbf{not integer} in the answer box if the answer is not an integer.

$122\div 2=\answer{\sage{122/2}}$
\end{problem}



\begin{problem}
Please, write \textbf{NI} for \textbf{not integer} in the answer box if the answer is not an integer.

$27\div 4=\answer{NI}$
\end{problem}


\begin{problem}
Please, write \textbf{NI} for \textbf{not integer} in the answer box if the answer is not an integer.

$23\cdot 12=\answer{\sage{23*12}}$
\end{problem}

\begin{problem}
Please, write \textbf{NI} for \textbf{not integer} in the answer box if the answer is not an integer.

$3\cdot (-5)=\answer{\sage{3*(-5)}}$
\end{problem}


\begin{problem}
Please, write \textbf{NI} for \textbf{not integer} in the answer box if the answer is not an integer.

$(-12)\cdot(-2)=\answer{\sage{-12*(-2)}}$
\end{problem}


\begin{problem}
Please, write \textbf{NI} for \textbf{not integer} in the answer box if the answer is not an integer.

$-(-5)=(-1)\cdot(-5)=\answer{\sage{-(-5)}}$
\end{problem}


\begin{problem}
Please, write \textbf{NI} for \textbf{not integer} in the answer box if the answer is not an integer.

$-(-(-3))=(-1)(-1)(-3)=\answer{\sage{(-1)*(-1)*(-3)}}$
\end{problem}




\begin{problem}
\textbf{True or False:} $0\div 3= 0$ 
\begin{multipleChoice*}
    \choice[correct]{True}
    \choice{False}
\end{multipleChoice*}
\end{problem}



\begin{problem}
\textbf{True or False:} $5\div 0= 0$ 
\begin{multipleChoice*}
    \choice{True}
    \choice[correct]{False}
\end{multipleChoice*}
\end{problem}


\begin{problem}
\textbf{True or False:} $0\div 0= 0$ 
\begin{multipleChoice*}
    \choice{True}
    \choice[correct]{False}
\end{multipleChoice*}
\end{problem}


\begin{problem}
Which law is used to get $$-3+5-6=(-3)+5+(-6)=(5)+(-3)+(-6)=5-3-6.$$ 
\begin{multipleChoice}
    \choice[correct]{Commutative Law}
    \choice{Associative Law}
    \choice{Identity Law}
\end{multipleChoice}
\end{problem}


\begin{problem}
Which law is used to get $$-3+5-6=2-6$$ 
\begin{multipleChoice}
    \choice{Commutative Law}
    \choice[correct]{Associative Law}
    \choice{Identity Law}
\end{multipleChoice}
\end{problem}


\begin{problem}
Which law is used to get $$30-30=(30)+(-30)=0$$ 
\begin{multipleChoice}
    \choice{Commutative Law}
    \choice{Associative Law}
    \choice[correct]{Identity Law}
\end{multipleChoice}
\end{problem}


\begin{problem}
Which law is used to get $$20\times 30=30\times 20$$ 
\begin{multipleChoice}
    \choice[correct]{Commutative Law}
    \choice{Associative Law}
    \choice{Identity Law}
\end{multipleChoice}
\end{problem}


\begin{problem}
Which law is used to get $$30\times 3 \times 7=30\times 21$$ 
\begin{multipleChoice}
    \choice{Commutative Law}
    \choice[correct]{Associative Law}
    \choice{Identity Law}
\end{multipleChoice}
\end{problem}



\begin{problem}
Write a number that is a whole number but not a natural number?

Answer is $\answer{0}$.
\end{problem}




\begin{problem}
Additive inverse of 3 is $\answer{-3}$.
\end{problem}




\begin{problem}
Additive inverse of -5 is $\answer{5}$.
\end{problem}

 
\begin{problem}
Additive inverse of $-(-5)$ is $\answer{-5}$.
\end{problem}

\begin{problem}
Additive inverse of $x$ is $\answer{-x}$.
\end{problem}

\begin{problem}
Additive inverse of $-a$ is $\answer{a}$.
\end{problem}

\begin{problem}

\begin{image}
\begin{tikzpicture}[x=0.75cm,>=stealth]
        \draw[<->] (-5,0)--(5,0);
        \foreach \x in {-4,...,4}
        \draw[shift={(\x,0)},color=black] (0pt,2pt) -- (0pt,-2pt); %node[below] {\footnotesize $\x$};
        \node[below] at (-5,-5pt) {$\ldots$};
        \node[below] at (5,-5pt) {$\ldots$};
        \fill (3,0) circle (2pt);
        \fill (-4,0) circle (2pt);
        \draw[<-,out=45,in=135,color=blue] (-4,0) to (-3,0) to (-2,0) to (-1,0) to (0,0) to (1,0) to (2,0) to (3,0);
        \node[color=red] at (3,-0.75) {\small Start};
        \node at (3,-0.3) {\small $3$};
        \node[color=blue] at (-4,-0.75) {\small End};
        \node at (-4,-0.3) {\small $b$};
        \node at (-0.5,0.5) {\small Move 7 units to the \emph{left}};
        %\draw[->,shorten >=5pt,shorten <=5pt,out=45,in=135,distance=0.5cm] (0,0) to (1,0);
\end{tikzpicture}
\end{image}

$$b=\answer{-4}.$$
\end{problem}

\begin{problem}

\begin{image}
\begin{tikzpicture}[x=0.75cm,>=stealth]
        \draw[<->] (-7,0)--(7,0); 
        \foreach \x in {-6,...,6}
%        \draw[shift={(\x,0)},color=black] (0pt,2pt) -- (0pt,-2pt); %node[below] {\footnotesize $\x$};
        \node[below] at (-7,-5pt) {$\ldots$};
        \node[below] at (7,-5pt) {$\ldots$};
        \fill (3,0) circle (2pt);
        \fill (-6,0) circle (2pt);
        \draw[->,out=15,in=165,color=blue] (-6,0) to (3,0);
        \node[color=red] at (3,-0.75) {\small End};
        \node at (3,-0.3) {\small $3$};
        \node[color=blue] at (-6,-0.75) {\small Start};
        \node at (-6,-0.3) {\small $-6$};
        \node at (-1.5,1) {\small Move \textbf{k} units to the \emph{right}};
        %\draw[->,shorten >=5pt,shorten <=5pt,out=45,in=135,distance=0.5cm] (0,0) to (1,0);
\end{tikzpicture}
\end{image}

$$k=\answer{9}.$$
\end{problem}

\begin{problem}
Jack has \$1500. He spends \$550 on a computer. He has \$$\answer{950}$ left.
\end{problem}


\begin{problem}
Jackie weighed 153 lbs a year ago. He put on 33 lbs since then. She now weighs $\answer{186}$ lbs.
\end{problem}


\begin{problem}
George invested \$1500 in stock market last month. He now has \$$1236$. His net gain is $\answer{\sage{1236-1500}}$ dollars.
\end{problem}

%\end{shuffle}

\end{document}
