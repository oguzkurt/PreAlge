\documentclass{ximera}
\pagestyle{empty}
\def\mystrut(#1,#2){\vrule height #1pt depth #2pt width 0pt}
\linespread{1.6}
\begin{document}

\noindent Solve by substitution:
\begin{shuffle}
\begin{question}
\begin{sagesilent}
var('x,y,z,t')
#Solution will be (x1,y1)
x1 = Integer(randint(1,6))
y1 = Integer(randint(1,6))
m1 = Integer(randint(1,6))
m2 = -Integer(randint(1,6)) 
a1 = Integer(randint(1,6))
a2 = a1 + Integer(randint(1,8))
\end{sagesilent}

$\begin{cases}
\sage{-m1*a1}x+\sage{a1}y=\sage{a1*y1-m1*x1*a1}\\
\sage{-m2*a2}x+\sage{a2}y=\sage{a2*y1-m2*x1*a2}\\
\end{cases}$\\\\

{\bf Solution:} The solution to the system is 
\answer{$(\sage{x1},\sage{y1})$}.\\\\
\end{question}

\begin{question}
\begin{sagesilent}
var('x,y,z,t')
#Solution will be (x1,y1)
x1 = Integer(randint(1,6))
y1 = Integer(randint(1,6))
m1 = Integer(randint(1,6))
m2 = -Integer(randint(1,6)) 
a1 = Integer(randint(1,6))
a2 = a1 + Integer(randint(1,8))
\end{sagesilent}
$\begin{cases}
\sage{-m1*a1}x+\sage{a1}y=\sage{a1*y1-m1*x1*a1}\\
\sage{-m2*a2}x+\sage{a2}y=\sage{a2*y1-m2*x1*a2}\\
\end{cases}$\\\\

{\bf Solution:} The solution to the system is 
\answer{$(\sage{x1},\sage{y1})$}.\\\\
\end{question}

\begin{question}
\begin{sagesilent}
var('x,y,z,t')
#Solution will be (x1,y1)
x1 = Integer(randint(1,6))
y1 = Integer(randint(1,6))
m1 = Integer(randint(1,6))
m2 = -Integer(randint(1,6)) 
a1 = Integer(randint(1,6))
a2 = a1 + Integer(randint(1,8))
\end{sagesilent}
$\begin{cases}
\sage{-m1*a1}x+\sage{a1}y=\sage{a1*y1-m1*x1*a1}\\
\sage{-m2*a2}x+\sage{a2}y=\sage{a2*y1-m2*x1*a2}\\
\end{cases}$\\\\

{\bf Solution:} The solution to the system is 
\answer{$(\sage{x1},\sage{y1})$}.\\\\
\end{question}
\begin{question}
\begin{sagesilent}
var('x,y,z,t')
#Solution will be (x1,y1)
x1 = Integer(randint(1,6))
y1 = Integer(randint(1,6))
m1 = Integer(randint(1,6))
m2 = -Integer(randint(1,6)) 
a1 = Integer(randint(1,6))
a2 = a1 + Integer(randint(1,8))
\end{sagesilent}
$\begin{cases}
\sage{-m1*a1}x+\sage{a1}y=\sage{a1*y1-m1*x1*a1}\\
\sage{-m2*a2}x+\sage{a2}y=\sage{a2*y1-m2*x1*a2}\\
\end{cases}$\\\\

{\bf Solution:} The solution to the system is 
\answer{$(\sage{x1},\sage{y1})$}.\\\\
\end{question}
\begin{question}
\begin{sagesilent}
var('x,y,z,t')
#Solution will be (x1,y1)
x1 = Integer(randint(1,6))
y1 = Integer(randint(1,6))
m1 = Integer(randint(1,6))
m2 = -Integer(randint(1,6)) 
a1 = Integer(randint(1,6))
a2 = a1 + Integer(randint(1,8))
\end{sagesilent}
$\begin{cases}
\sage{-m1*a1}x+\sage{a1}y=\sage{a1*y1-m1*x1*a1}\\
\sage{-m2*a2}x+\sage{a2}y=\sage{a2*y1-m2*x1*a2}\\
\end{cases}$\\\\

{\bf Solution:} The solution to the system is 
\answer{$(\sage{x1},\sage{y1})$}.\\\\
\end{question}
\begin{question}
\begin{sagesilent}
var('x,y,z,t')
#Solution will be (x1,y1)
x1 = Integer(randint(1,6))
y1 = Integer(randint(1,6))
m1 = Integer(randint(1,6))
m2 = -Integer(randint(1,6)) 
a1 = Integer(randint(1,6))
a2 = a1 + Integer(randint(1,8))
\end{sagesilent}
$\begin{cases}
\sage{-m1*a1}x+\sage{a1}y=\sage{a1*y1-m1*x1*a1}\\
\sage{-m2*a2}x+\sage{a2}y=\sage{a2*y1-m2*x1*a2}\\
\end{cases}$\\\\

{\bf Solution:} The solution to the system is 
\answer{$(\sage{x1},\sage{y1})$}.\\\\
\end{question}

\end{shuffle}


Solve by elimination:
\begin{sagesilent}
var('x,y,z,t')
#Solution will be infinite (parallel lines)
x2 = Integer(randint(1,8))
y2 = Integer(randint(1,8))
m = Integer(randint(3,8))
c2 = Integer(randint(1,6))
\end{sagesilent}
$\begin{cases}
\sage{x2}x+\sage{y2}y=\sage{c2}\\
\sage{m*x2}x+\sage{m*y2}y=\sage{m*c2}\\
\end{cases}$\\\\
{\bf Solution:} Since the second equation is obtained from the first equation
by multiplying by $\sage{m}$ the lines are parallel. Therefore, the solution
set is infinite and can be expressed using the free variable $y$ as
$\{(\sage{(c2-y2*y)/x2}, y):y \in \mathbb{R}\}$.\\\\


Solve by elimination:
\begin{sagesilent}
var('x,y,z')
#Solution will be infinite (parallel lines)
x2 = Integer(randint(1,8))
y2 = Integer(randint(1,8))
m = Integer(randint(3,8))
m2 = m+Integer(randint(3,8))
c2 = Integer(randint(1,6))
\end{sagesilent}
$\begin{cases}
\sage{x2}x+\sage{y2}y=\sage{c2}\\
\sage{m*x2}x+\sage{m*y2}y=\sage{m2*c2}\\
\end{cases}$\\\\
{\bf Solution:} Take the second equation and subtract $\sage{m}$ times the
first equation will result in the second equation being $0=\sage{(m2-m)*c2}$.
This means the system is inconsistent; there is no solution.\\\\
\end{document}





\begin{sagesilent}
A = random_matrix(ZZ,3,3)
while A.determinant() == 0:
    A = random_matrix(ZZ,3,3)
I2 = identity_matrix(2)
B = random_matrix(ZZ,2,2)
while B.determinant() == 0:
    B = random_matrix(ZZ,2,2)
\end{sagesilent}
\noindent Find the inverse of the matrix $A=\sage{A}$.\\\\
\noindent \mystrut(5,8) Use elementary row operations on the augmented
marix\\$\sage{B.augment(I2,subdivide=True)}$ to find the inverse of $\sage{B}$.
Show your steps!\\\\\\
{\bf Solution:} The inverse of   $\sage{A}$ is \answer{$\sage{A.inverse()}$}.\\\\\\
{\bf Solution:} The inverse of   $\sage{B}$ is  \answer{$\sage{B.inverse()}$}.

%\end{document}
